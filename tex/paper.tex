\documentclass[10pt,twocolumn,letterpaper]{article}

\usepackage{cvpr}
\usepackage{times}
\usepackage{epsfig}
\usepackage{graphicx}
\usepackage{amsmath}
\usepackage{amssymb}

% Include other packages here, before hyperref.

% If you comment hyperref and then uncomment it, you should delete
% egpaper.aux before re-running latex.  (Or just hit 'q' on the first latex
% run, let it finish, and you should be clear).
\usepackage[breaklinks=true,bookmarks=false]{hyperref}

\cvprfinalcopy % *** Uncomment this line for the final submission

\def\cvprPaperID{****} % *** Enter the CVPR Paper ID here
\def\httilde{\mbox{\tt\raisebox{-.5ex}{\symbol{126}}}}

% Pages are numbered in submission mode, and unnumbered in camera-ready
%\ifcvprfinal\pagestyle{empty}\fi
\begin{document}

%%%%%%%%% TITLE
\title{Object Detection and Recognition: A Survey}

\author{Reymundo A. Gutierrez\\
Massachusetts Institute of Technology\\
{\tt\small ragtz@mit.edu}\\
% For a paper whose authors are all at the same institution,
% omit the following lines up until the closing ``}''.
% Additional authors and addresses can be added with ``\and'',
% just like the second author.
% To save space, use either the email address or home page, not both
\and
Juli\'{a}n A. Gonz\'{a}lez\\
Massachusetts Institute of Technology\\
{\tt\small jugonz97@mit.edu}
}

\maketitle
%\thispagestyle{empty}

%%%%%%%%% ABSTRACT
\begin{abstract}
   Should we put an abstract here?
\end{abstract}

%%%%%%%%% BODY TEXT
\section{Introduction}

Object recognition, the problem of mapping an image to a class label, is one of the core
problems in computer vision.
A related fundamental problem is object detection, also commonly referred to as object recognition.
(To avoid ambiguity, here we will define object detection as detecting and localizing objects and
performing object recognition).
The applications of object recognition and detection are endless, from simple visual search of
images to high-level scene understanding. It is by no means a solved problem.
This paper provides a survey of the current landscape of approaches toward the problem,
laying a conceptual foundation for the implementation of a convolutional neural net for
digit recognition.

Modern approaches towards object recognition differ significantly but tend to depend on
properties of images (features) that are not disturbed during transformations.
As an example, \cite{LoweObjSIFT} informed the computer vision community about the importance of local
scale-invariant features on images.
These lower-level features are still useful in innovative systems, but because of their
locality are hard to utilize alone.
Relationships between features can be leveraged to determine higher-level structure of
objects with careful estimation \cite{PartModels}.
Higher-level structure can be thought of as a hierarchy, which allows for composition of features \cite{HintonDBN}.
With such a hierarchy it is possible to look for translation-invariant features \cite{CDBN}.
This demonstrates that the difficulty of object recognition often lies in determining
appropriate representations of image data.

The rest of this paper is organized as follows.
Section 2 provides an overview of several modern approaches to object recognition
roughly in chronological order, to make it easier to see how strikingly different
solutions to the same problem have evolved.
Section 3 shows how we will use ideas from these approaches to construct
a convolutional neural network for recognizing hand-written digit data.

%------------------------------------------------------------------------
\section{Review of Methods}
\subsection{Scale-Invariant Feature Transform}

One way to begin to perform object detection / recognition is to identify discriminating
features of the object (i.e. features that can be mapped to a specific object or class of
objects with high probability). It is also necessary for robustness that these features
possess a certain degree of invariance to changes in translation, rotation, scale, and
illumination. The scale-invariant feature transform (SIFT) algorithm focuses on detecting
such features \cite{LoweObjSIFT}.

SIFT features are generated by first locating points that have translational, rotational,
and scale invariance by finding the maxima and minima of a difference of Gaussians
applied to an image pyramid. Each point is then assigned a canonical orientation
determined by the peak in a histogram of local image gradient orientations. To add
robustness to local geometric distortion, each point can be represented by a set of
images at different orientations and its encapsulated gradients at each orientation,
with intermediate gradients being determined through interpolation and blurred for
added translational invariance.

Object detection through SIFT features is fairly robust and can even function under
occlusion when many features are used to represent an object.

%-------------------------------------------------------------------------
\subsection{Part Based Models}

\cite{PartModels} models objects as collections of smaller parts.
In a part-based model, object parts are connected to each other via springs that
allow constrained deformation. This allows the construction of part-specific filters.
In addition, a root filter that is tuned for the entire object operates at half
the spatial resolution as the part filters, allowing for multi-spatial recognition.
Object detection in this scheme uses the best set of part filter responses given a
location for the root filter.

Because the training data used only contains bounding boxes for objects,
part locations must be learned. Training is done using a specialized latent support
vector machine that identifies part locations. Effort is also taken to locate a
number of difficult negative examples for specialized training.

Part-based models can be computed efficiently: dynamic programming algorithms
can be used to calculate part locations and the system in \cite{PartModels} can be trained
and run on a real dataset in less than 8 hours.
This system however has not aged well: the canonical implementation's average precision
scores on the PASCAL VOC 2007 \cite{PascalVOC} test have been eclipsed in recent years
by convolutional neural nets.

%-------------------------------------------------------------------------
\subsection{Deep Belief Networks}

While the methods discussed previously explicitly describe the features to be used
by the representation, deep belief nets (DBN) allow for the discovery of such features
from a training set \cite{HintonDBN} \cite{CDBN}.
DBNs, like parts-based models, also look at the relationship between features of an object,
but do not constrain the type of relation.
Instead, DBNs build a hierarchical representation of the object with each level of the hierarchy
representing more abstract features.

The basic component of a DBN is the restricted Boltzmann machine (RBM).
An RBM is a bipartite, undirected graphical model that can learn a probability distribution
over its inputs and whose edges are represented by a weight matrix.
Being bipartite, the RBM forms two layers; the layer associated with the input is known
as the visible layer and the other as the hidden layer.
Following the probabilistic semantics laid out in \cite{CDBN}, the weight parameters of the RBM
are optimized by performing gradient-based contrastive divergence, which utilizes a
Gibbs sampling step on the conditional probability of the hidden and visible units given
the visible and hidden units respectively.
These conditional probabilities are known as the output function, which often take the form of a sigmoid.
The DBN is then constructed by stacking RBMs into several layers.
The whole network is trained by greedily training each layer in succession up the hierarchy,
freezing the parameters of each previous layer.
To produce class labels, either the labels are included as input or a classification algorithm
can be trained on the higher level features.

DBNs are successful on classification tasks in certain domains, such as the MNIST digit recognition
task \cite{HintonDBN}. Unfortunately, they are difficult to scale to realistically sized images
due to the representational redundancy needed to compensate for their loss of image
two dimensional structure. This loss also makes translational invariance difficult to achieve.

%-------------------------------------------------------------------------
\subsection{Convolutional Deep Belief Networks}

To address the scalability and translational invariance issues inherent in DBNs,
Lee et al. introduce the convolutional deep belief network (CDBN),
which augments DBNs with convolutional restricted Boltzmann machines (CRBMs)
and probabilistic max-pooling layers between CRBMs \cite{CDBN}.
Like RBMs, CRBMs consist of a visible layer and hidden layer, but the hidden
layer here is partitioned into \(K \) square blocks, each associated with a filter
whose weights are shared amongst the units in a block.
The relation between the visible units and the corresponding hidden units in each
block is defined by a convolution of the weight matrix and the image.

To alleviate the scaling issue, a probabilistic max-pooling layer is added above the CRBM,
forming a max-pooling CRBM.
This shrinks the representation of the hidden layer by a constant factor.
These pooling layers contain \(K \) blocks, with each block mapping to the corresponding hidden
layer block below it as follows: the hidden layer block is partitioned into square blocks with the
corresponding pooling unit for each block being set to 1 if and only if a unit in its hidden layer
block is 1, of which only one can be non-zero.
The CDBN is constructed by stacking max-pooling CRBMs,
which are trained similarly to RBMs.

The training of CDBNs is accomplished using the same greedy approach as in DBNs.
Once the parameters have been learned, the CDBNs representation of an image can be computed
by sampling from the joint distribution over all the hidden layers given the input image.
The classification of images can be produced by training a classification algorithm
on the higher level features.

Because CDBNs construct a hierarchical representation of their input that is robust
to appearance, scale, and occlusion variation, they are are able to deliver
state-of-the-art performance.
They are not without shortcomings: in particular, they suffer from overfitting due
to overrepresentation of their input data.
There are several solutions to this problem. In dropout, the output of a hidden layer
unit is set to 0 with probability \(1/2 \); this unfortunately significantly increases training
time (which can already take weeks). Another common solution, data augmentation,
is to insert altered forms of the input images into the dataset (used in \cite{Verydeep}).

Modifications to the original CDBN architecture \cite{CDBN} have produced the most performant
systems to date.
By overlapping hidden layer blocks during max-pooling, \cite{ImageNet} increases performance.
The approach outlined in \cite{Verydeep} gains significant accuracy by adding many extra layers
to the CDBN and emphasizing very small filters.

In order to counter the extra complexity introduced by these changes,
\cite{ImageNet} and \cite{Verydeep} replace the standard sigmoid nonlinear output function with the
Rectified Linear Unit (ReLU) function \(f(x) = max(0, x) \) for quicker training.
Additionally, both use multiple GPUs for significant speedup in matrix operations
and easy parallelism.
Nevertheless, these CDBNs take a very long time to train: the net in \cite{Verydeep}
takes 2-3 weeks to train using four GPUs.

%-------------------------------------------------------------------------
\section{MNIST Digit Recognition Proposal}

Utilizing some of the ideas outlined for object recognition in section 2,
we will implement a convolutional deep belief network in the style of \cite{CDBN} to recognize
handwritten digits from the MNIST database \cite{MNIST}.
This CDBN will determine image features to train an off-the-shelf support-vector machine
that will perform the classification step.
It is clear that properties of the net like the number of layers, the pooling procedure,
and choice of non-linear output function will require experimentation to optimize.
Currently we believe that a net as deep as \cite{Verydeep} may not be necessary or may even perform worse than a
somewhat shallower net.
We plan on implementing max-pooling in a way that allows us to vary the overlap of blocks in the
procedure, including zero overlap.
Finally, we plan on evaluating our net using both the sigmoid and ReLU nonlinear
output functions to test the tradeoff in training time and performance.

Because handwritten digits share many properties with handwritten characters
and highway indicator signs, we feel it is natural to evaluate our CDBN on these datasets.
We will test this theory on our net with data from REMOVED and REMOVED
and compare its effectiveness relative to recognizing digits.

{\small
\bibliographystyle{ieee}
\bibliography{bib}
}

\end{document}
